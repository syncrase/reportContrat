\usepackage[utf8]{inputenc}
\usepackage[T1]{fontenc}
%\usepackage{lipsum}% juste utile ici pour générer du faux texte}
\usepackage{mwe}%juste utile ici pour générer de fausses images
\usepackage{amsmath,amsfonts,amssymb}%extensions de l'ams pour les mathé matiques
\usepackage{shorttoc}%pour la réalisation d'un sommaire.
\usepackage{tikz}
\usepackage{graphicx}%pour insérer images et pdf entre autres
	\graphicspath{{images/}}%pour spécifier le chemin d'accès aux images
\usepackage{pdfpages}
\usepackage{geometry}%réglages des marges du document selon vos préférences ou celles de votre établissemant[left=3.5cm,right=2.5cm,top=4cm,bottom=4cm]
\usepackage[Lenny]{fncychap}%pour de jolis titres de chapitres voir la doc pour d'autres styles.

\usepackage{fancyhdr}%pour les entêtes et pieds de pages
	\setlength{\headheight}{14.2pt}% hauteur de l'entête

%%%%%%%%%%%%%%%%%%%style front%%%%%%%%%%%%%%%%%%%%%%%%%%%%%%%%%%%%%%%%%	
	\fancypagestyle{front}{%
  		\fancyhf{}%on vide les entêtes
  		\fancyfoot[C]{page \thepage}%
  		\renewcommand{\headrulewidth}{0pt}%trait horizontal pour l'entête
  		\renewcommand{\footrulewidth}{0.4pt}%trait horizontal pour les pieds de pages
		}


%%%%%%%%%%%%%%%%%%%style main%%%%%%%%%%%%%%%%%%%%%%%%%%%%%%%%%%%%
	\fancypagestyle{main}{%
		\fancyhf{}
  		\renewcommand{\chaptermark}[1]{\markboth{\chaptername\ \thechapter.\ ##1}{}}% redéfintion pour avoir ici les titres des chapitres des sections en minuscules
  		\renewcommand{\sectionmark}[1]{\markright{\thesection\ ##1}}
		\fancyhead[c]{}
		\fancyhead[RO,LE]{\rightmark}%
  		\fancyhead[LO,RE]{\leftmark}
		\fancyfoot[C]{}
		\fancyfoot[RO,LE]{page \thepage}%
  		%\fancyfoot[LO,RE]{Rapport de contrat de professionnalisation}
  		}

%%%%%%%%%%%%%%%%%%%style back%%%%%%%%%%%%%%%%%%%%%%%%%%%%%%%%%%%%%%%%%	
	\fancypagestyle{back}{%
  		\fancyhf{}%on vide les entêtes
  		\fancyfoot[C]{page \thepage}%
  		\renewcommand{\headrulewidth}{0pt}%trait horizontal pour l'entête
  		\renewcommand{\footrulewidth}{0.4pt}%trait horizontal pour les pieds de pages
		}

%%%%%%%%%%%%%%%%%%%style abstract%%%%%%%%%%%%%%%%%%%%%%%%%%%%%%%%%%%%%%%%%	
\makeatletter
\newenvironment{abstract}{%
    \cleardoublepage
    \null\vfil
    \@beginparpenalty\@lowpenalty
    \begin{center}%
      \bfseries \abstractname
      \@endparpenalty\@M
    \end{center}}%
   {\par\vfil\null}
\makeatother

%%%%%%%%%%%%%%%%%%%%%%%%%%%%index%%%%%%%%%%%%%%%%%%%%%%%%%%%%%%%%%%%%%%%
\usepackage{makeidx}
%\usepackage{xindy}
\makeindex




%\usepackage[thinlines]{easytable}
%\usepackage{changepage}
%\usepackage{fullwidth}


%\DeclareUnicodeCharacter{00A0}{~}



%\usepackage{newunicodechar}
%\newunicodechar{^^a0}{~}




\usepackage{textcomp}%  \texttildelow




%\usepackage[demo]{graphicx}% delete the demo option in your actual code
\usepackage{enumitem}
\usepackage{booktabs}

\usepackage{float}


\usepackage[english,french]{babel}%pour un document en français
\usepackage{hyperref}%rend actif les liens, références croisée, toc, ...
		\hypersetup{colorlinks,%
		citecolor=black,%
		filecolor=black,%
		linkcolor=black,%
		urlcolor=black} 
%%%%%%%%%%%%%%%%%%%%%%%%%%%%biblio%%%%%%%%%%%%%%%%%%%%%%%%%%%%%%%%%%%%%%
%% run bibtex report.aux
\usepackage[backend=bibtex]{biblatex}
\addbibresource{bibliographie/biblio.bib}% pour indiquer ou se trouve notre .bib
%\bibliography{bibliographie/biblio}
\usepackage{csquotes}% pour la gestion des guillemets français.

%%%%%%%%%%%%%%%%%%%%%%%%%%%%%glossaire%%%%%%%%%%%%%%%%%%%%%%%%%%%%%%%%%%%
%% run pdflatex report.tex
%% depuis le répertoire courant : run makeglossaries report.glo
%% run 2x  pdflatex report.tex

%Les entrées au glossaire se font grâce aux commandes
% \gls{latex}  \\
%\glspl{lvm}

\usepackage{glossaries}
\makeglossaries		

%%%%%%%%%%%%%%%%%%%%%%%%%%%%liste des abbréviations%%%%%%%%%%%%%%		
%\usepackage[french]{nomencl}
%\makenomenclature
%\renewcommand{\nomname}{Liste des abréviation, des sigles et des symboles}

%\glsresetall
%\glsaddall


%Please load the package listings after 'babel'
\usepackage{listings}
\usepackage{color}
\definecolor{mygreen}{rgb}{0,0.6,0}
\definecolor{mygray}{rgb}{0.5,0.5,0.5}
\definecolor{mymauve}{rgb}{0.58,0,0.82}
\lstset{ %
  backgroundcolor=\color{white},   % choose the background color; you must add \usepackage{color} or \usepackage{xcolor}
  basicstyle={\small\ttfamily},        % \footnotesize : the size of the fonts that are used for the code
  breakatwhitespace=false,         % sets if automatic breaks should only happen at whitespace
  breaklines=true,                 % sets automatic line breaking
  captionpos=none,                    % b : sets the caption-position to bottom
  commentstyle=\color{mygreen},    % comment style
  deletekeywords={...},            % if you want to delete keywords from the given language
  escapeinside={\%*}{*)},          % if you want to add LaTeX within your code
  extendedchars=true,              % lets you use non-ASCII characters; for 8-bits encodings only, does not work with UTF-8
  frame=none,	                   % single : adds a frame around the code
	% tb : 
  keepspaces=true,                 % keeps spaces in text, useful for keeping indentation of code (possibly needs columns=flexible)
	columns=flexible,
  keywordstyle=\color{blue},       % keyword style
  language=Java,                 % the language of the code
  otherkeywords={*,...},            % if you want to add more keywords to the set
  numbers=left,                    % where to put the line-numbers; possible values are (none, left, right)
  numbersep=5pt,                   % how far the line-numbers are from the code
  numberstyle=\tiny\color{mygray}, % the style that is used for the line-numbers
  rulecolor=\color{black},         % if not set, the frame-color may be changed on line-breaks within not-black text (e.g. comments (green here))
  showspaces=false,                % show spaces everywhere adding particular underscores; it overrides 'showstringspaces' <=> showstringspaces=false
  showstringspaces=false,          % underline spaces within strings only
  showtabs=false,                  % show tabs within strings adding particular underscores
  stepnumber=2,                    % the step between two line-numbers. If it's 1, each line will be numbered
  stringstyle=\color{mymauve},     % string literal style
  tabsize=2,	                   % sets default tabsize to 2 spaces
  title=\lstname,                   % show the filename of files included with \lstinputlisting; also try caption instead of title
	aboveskip=3mm,
  belowskip=3mm
}





