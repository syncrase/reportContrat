\chapter*{Bilan}
\addcontentsline{toc}{chapter}{Bilan}
\markboth{Bilan}{}

Les diff\'{e}rentes missions qu'il m'a \'{e}t\'{e} donn\'{e} d'effectuer m'ont d'abord permis de participer activement \`{a} l'activit\'{e} de l'entreprise et de me retrouver dans la peau d'un ing\'{e}nieur qualit\'{e}, me posant des questions n\'{e}cessitant une vrai compr\'{e}hension du produit et du recul par rapport \`{a} mes objectifs.\\

Lorsque mes coll\`{e}gues n'avaient pas le temps de s'occuper de ce qu'ils voulaient, je me suis vu affecter leurs t\^{a}ches. Me pencher sur le genre de travaux que l'on donne aux ing\'{e}nieurs a \'{e}t\'{e} pour moi une marque de reconnaissance. Mais paradoxalement cela a aussi \'{e}t\'{e} \`{a} l'origine de remises en question \textquote{En serai-je capable?}, \textquote{Ai-je les connaissances requises?}, \textquote{Vais-je r\'{e}ussir dans le temps imparti?}.\\

Le d\'{e}veloppement d'un plugin Jenkins a \'{e}t\'{e} une vraie \'{e}tapes dans ma carri\`{e}re de d'ing\'{e}nieur, un pied de pos\'{e} dans la r\'{e}alit\'{e} du d\'{e}veloppement logiciel mais surtout du logiciel libre. Signifiant que j'ai particip\'{e} activement \`{a} l'\'{e}volution de ce logiciel open-source et communautaire. J'ai communiqu\'{e} avec le concepteur de l'ancien plugin, que le mien remplace, et c'est avec plaisir qu'il m'a apport\'{e} les r\'{e}ponses \`{a} mes questions. Mon plugin n'est, pour l'heure, pas encore sur le repository officiel mais cela est en discussion. Cette \'{e}ventualit\'{e} permettrai \`{a} n'importe quel utilisateur de Jenkins d'essayer mon plugin et de l'adopter, ou de le rejeter. Bien \'{e}videmment, le rendre disponible publiquement le mettrais \`{a} l'\'{e}preuve d'utilisation pour lesquelles il n'avait pas \'{e}t\'{e} pr\'{e}vu initialement. Il est donc certain que des bugs seront trouv\'{e}s, autrement dit, le livrer sur le repository officiel n\'{e}cessite de le maintenir et de rester disponible pour la maintenance, l'\'{e}volution et pour r\'{e}pondre aux questions des utilisateurs.\\

Techniquement, j'ai eu la chance de pouvoir me pencher sur plusieurs sujets diff\'{e}rents, tenant de domaines techniques diff\'{e}rents et \`{a} des fins diff\'{e}rentes. En d\'{e}veloppant les tests j'ai impl\'{e}ment\'{e} du code Java, non pour apprendre une technologie, mais pour r\'{e}ussir \`{a} faire quelque chose, me focalisant ainsi sur l'objectif et les performances plut\^{o}t que sur la mani\`{e}re.\\

De tout ce que j'ai pu faire lors de ce contrat, seule une petite partie de ce que j'ai appris \`{a} l'\'{e}cole m'a \'{e}t\'{e} n\'{e}cessaire, bien qu'essentielle. Je me suis vu aborder des probl\'{e}matiques et des technologies dont j'avais, tout au plus, vaguement entendu parl\'{e}. J'ai donc adopt\'{e} une vrai m\'{e}thode de travail o\`{u} ce qui compte n'est pas la maitrise de telle ou telle technologie, comme en \'{e}cole d'ing\'{e}nieur, mais de proposer une solution fiable et performante, quelle que soit le moyen utilis\'{e}. Ce faisant je me suis pos\'{e} les vraies questions qui concernent v\'{e}ritablement un ing\'{e}nieur : \textquote{Quels sont les diff\'{e}rents moyens de r\'{e}soudre ce probl\`{e}me?}, \textquote{Quels sont les diff\'{e}rents avantages et inconv\'{e}nients?}, \textquote{Quelles sont les normes, les n\'{e}cessit\'{e}s, les limitions juridiques?}, \textquote{Quelles sont les personnes comp\'{e}tentes qui pourrons me donner des informations primordiales pour mon projet?}.\\






D'un point de vue personnel, ce qui a \'{e}t\'{e} un frein au d\'{e}but de mon contrat de professionnalisation f\^{u}t, d'une part, mon acharnement \`{a} vouloir tout comprendre dans les d\'{e}tails sans que cela soit n\'{e}cessaire, et d'autre part, \`{a} ne pas me tourner vers les personnes comp\'{e}tentes qui auraient pu me conseiller, r\'{e}duisant ainsi des p\'{e}riodes d'\'{e}tudes de plusieurs jours \`{a} quelques heures. Je regrette de ne pas avoir \'{e}t\'{e} suffisamment demandeur d'informations, d\'{e}sirant faire le mieux possible, le plus vite et sans soutien ce que l'on attendait de moi, ainsi, je n'ai pas appris tout ce que j'aurai pu.\\

Si c'\'{e}tait \`{a} refaire je ne me jetterai pas tout suite dans l'\'{e}laboration de ma strat\'{e}gie pour arriver au bout de ma mission, mais passerai beaucoup plus de temps \`{a} me diriger vers les autres pour discuter des contours de mon projets, des probl\`{e}mes potentiels et des solutions d\'{e}j\`{a} existentes; chose que je faisait beauxoup plus facilement plus tard, mon contrat arrivant \`{a} son terme.\\


Le temps que j'ai pass\'{e} dans cette entreprise est une immense fiert\'{e} dont je n'h\'{e}siterai pas \`{a} vendre les m\'{e}rites. Parce que c'est dans un contexte internationnal que j'ai particip\'{e} au d\'{e}veloppement du framework de test, et parce que c'est pour le leader de la BI que j'ai d\'{e}velopp\'{e} un outil open-source, utilis\'{e} quotidiennement, de reporting des builds. Ce contrat de professionnalisation \`{a} \'{e}t\'{e} une exp\'{e}rience exceptionnellement riche, tant par ce que j'ai appris \`{a} faire ou \`{a} \^{e}tre, que par les nombreuses connaissances techniques que j'ai pu accumuler. 



