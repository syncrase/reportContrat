

\newglossaryentry{build}
{
	name={build},
	description={Anglicisme signifiant construction. Dans le contexte du développement logiciel la build est soit l'action de compiler le code source, soit le résultat de cette compilation.}
}
\newglossaryentry{plugin}
{
	name={plugin},
	description={extension d'un logiciel, celui-ci complète son fonctionnement est apporte de nouvelles fonctionnalités.}
}
\newglossaryentry{repository}
{
	name={repository},
	description={Anglicisme signifiant dépôt ou référentiel, c'est un espace de stockage centralisé et organisé des données}
}

\newglossaryentry{BOE}
{
	name={Business Objects Entreprise},
	description={Suite de logiciels comprenant Web Intelligence}
}
\newglossaryentry{Workflow}
{
	name={Workflow},
	description={Un workflow est une succession d'action permettant de partir d'un état A à un état B du logiciel}
}
\newglossaryentry{Client lourd}
{
	name={Client lourd},
	description={Un client lourd est un logiciel que l'on installe sur un machnie physique}
}
\newglossaryentry{Client léger}
{
	name={Client léger},
	description={Un client léger est une application hébergée sur serveur dont l'accès se fait via un navigateur (Internet Explorer, Google Chrome, Mozilla Firefox, etc.)}
}
\newglossaryentry{CMS}
{
	name={CMS},
	description={Le }
}
\newacronym{SDK}{SDK}{Software development kit}
\newacronym{ST}{ST}{Software tester}
\newacronym{WebI}{WebI}{Web Intelligence}
\newacronym{BI}{BI}{Business intelligence}
\newacronym{EPR}{ERP}{Entreprise Ressource Planning}
\newacronym{SAP}{SAP}{Systems Application and Products in data processing}

%\newacronym{}{}{}

%\newglossaryentry{}
%{
%	name={},
%	description={}
%}