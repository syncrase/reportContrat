

\newglossaryentry{build}
{
	name={build},
	description={Anglicisme signifiant construction. Dans le contexte du développement logiciel la build est soit l'action de compiler le code source, soit le résultat de cette compilation.}
}
\newglossaryentry{plugin}
{
	name={plugin},
	description={extension d'un logiciel, celui-ci complète son fonctionnement est apporte de nouvelles fonctionnalités.}
}
\newglossaryentry{repository}
{
	name={repository},
	description={Anglicisme signifiant dépôt ou référentiel, c'est un espace de stockage centralisé et organisé des données}
}

\newglossaryentry{BOE}
{
	name={Business Objects Entreprise},
	description={Suite de logiciels comprenant Web Intelligence}
}
\newglossaryentry{Workflow}
{
	name={Workflow},
	description={Un workflow est une succession d'action permettant de partir d'un état A à un état B du logiciel}
}
\newglossaryentry{Client lourd}
{
	name={client lourd},
	description={Un client lourd est un logiciel que l'on installe sur un machnie physique}
}
\newglossaryentry{Client léger}
{
	name={client léger},
	description={Un client léger est une application hébergée sur serveur dont l'accès se fait via un navigateur (Internet Explorer, Google Chrome, Mozilla Firefox, etc.)}
}
\newglossaryentry{CMS}
{
	name={CMS},
	description={Le }
}
\newglossaryentry{Eclipse}
{
	name={Eclipse},
	description={Eclipse est un IDE open source de développement Java}
}
\newglossaryentry{Classe}
{
	name={classe},
	description={Une classe, en programmation orientée objet, déclare un ensemble d'attributs et de méthodes communs à un ensemble d'objets}
}
\newglossaryentry{Attribut}
{
	name={attribut},
	description={Un attribut est une propriété typée caractéristique d'un objet}
}
\newglossaryentry{Methode}
{
	name={m\'{e}thode},
	description={Une méthode est un comportement propre à une classe}
}
\newglossaryentry{Testplan}
{
	name={plan de test},
	description={Un plan de test (ou testplan) est une collection de suite de test}
}
\newglossaryentry{Testsuite}
{
	name={suite de tests},
	description={Une test suite est l'ensemble des tests effectués lors de son exécution}
}
\newglossaryentry{Testcase}
{
	name={testcase},
	description={C'est un cas de test, correspond à l'implémentation d'une classe de test}
}

\newglossaryentry{Compilation}
{
	name={compilation},
	description={La compilation d'un programme c'est la traduction du code source lisible par l'homme en un autre langage compréhensible par la machine}
}

\newglossaryentry{Software tester}
{
	name={software tester},
	description={Ingénieur écrivant, entre autre, des tests automatiques pour s'assurer, d'une part, que les fonctionnalités correspondent bien aux spécifications, et d'autre part, permettre aux régressions d'être traitées rapidement}
}

\newglossaryentry{Interface}
{
	name={interface},
	description={Une interface définit le comportement d'une classe, concrètement une interface définit un ensemble de méthodes que doit nécéssairement implémenter une classe}
}

\newglossaryentry{Queryspec}
{
	name={queryspec},
	description={C'est une spécification de requête au format XML propre à Business Objects}
}

\newglossaryentry{Framework}
{
	name={framework},
	description={Un Framework est un environnement de développement comprenant une suite d'outils nécessaires au développement comme par exemple un IDE, des codes sources, des conventions de nommages, \ldots}
}
\newglossaryentry{Java}
{
	name={Java},
	description={Java est un langage de programmation orientée objet (sans être un langage purement objet, Java utilise les types primitifs int, char, float, etc.) développé par Oracle}
}

\newglossaryentry{API}
{
	name={API},
	description={Signifie Application Programmable Interface}
}

\newglossaryentry{Web service}
{
	name={web service},
	description={Un web service est une méthode que l'on appel grâce à une URL}
}


\newglossaryentry{JAR}
{
	name={JAR},
	description={Acronyme de Java ARchive}
}

\newglossaryentry{SDK}
{
	name={SDK},
	description={Acronyme de Software development kit}
}

\newglossaryentry{ST}
{
	name={ST},
	description={Acronyme de Software tester}
}

\newglossaryentry{BI}
{
	name={BI},
	description={Acronyme de Business intelligence}
}

\newglossaryentry{EPR}
{
	name={EPR},
	description={Acronyme de Entreprise Ressource Planning}
}
\newglossaryentry{SAP}
{
	name={SAP},
	description={Acronyme de Systems Application and Products in data processing en anglais et Systeme, Anwendungen und Produkte in der DatenverarbeitungSystems en allemand}
}
\newglossaryentry{SSL}
{
	name={SSL},
	description={Signifie Secure Sockets Layer, peut aussi se retrouver sous la dénomination TLS (Transport Layer Security)}
}
\newglossaryentry{JDK}
{
	name={JDK},
	description={Acronyme de Java Development Kit}
}
\newglossaryentry{Maven}
{
	name={Maven},
	description={Outil d'automatisation du processus de production logiciel, en l'occurrence la gestion des sources}
}

\newglossaryentry{Netbeans}
{
	name={Netbeans},
	description={Netbeans est un environnement de développement intégré}
}













%\newacronym{SDK}{SDK}{Software development kit}
%\newacronym{ST}{ST}{Software tester}
%\newacronym{WebI}{WebI}{Web Intelligence}
%\newacronym{BI}{BI}{Business intelligence}
%\newacronym{EPR}{ERP}{Entreprise Ressource Planning}
%\newacronym{SAP}{SAP}{Systems Application and Products in data processing}

%\newacronym{}{}{}

%\newglossaryentry{}
%{
%	name={},
%	description={}
%}