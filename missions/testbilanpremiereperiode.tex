\clearpage
\section{Bilan de la 1\up{\`{e}re} p\'{e}riode}
Tout au long des mois de juillet, ao\^{u}t et septembre j'ai impl\'{e}ment\'{e} de nombreux tests\footnote{cf. annexe \ref{pdf:ImplementedTestsList} page \pageref{pdf:ImplementedTestsList} pour la liste compl\`{e}te des tests impl\'{e}ment\'{e}s} pour des bugs divers, que ce soit une faute de frappe ou une fonctionnalit\'{e} ne fonctionnant plus du tout. J'ai beaucoup appris de mes erreurs, surtout lorsque plusieurs jours de travail s'av\'{e}raient inutiles parce qu'une meilleure solution, \'{e}vidente pour qui sait, existait.\\
J'ai aussi pu parfaire mes connaissances en Java ainsi que ma m\'{e}thodologie lorsque je dois me former \`{a} une nouvelle technologie.\\
Au d\'{e}but de cette mission, je n'avais jamais impl\'{e}ment\'{e} de tests \`{a} l'exception de tests unitaires. Aujourd'hui, je suis ravi de comprendre la probl\'{e}matique du test et quels types de tests existent. Aujourd'hui je sais pourquoi et comment les impl\'{e}menter et suis capable de faire des tests qui soient facilement maintenables.




\subsection{Connaissance du framework}
La principale perte de temps \'{e}tait d\^{u}e \`{a} une m\'{e}connaissance du Framework de test. Celui-ci poss\'{e}dant un grand nombre de m\'{e}thodes aux int\'{e}r\^{e}ts divers et vari\'{e}s, d'embl\'{e}e, il est tr\`{e}s difficile de pouvoir d\'{e}terminer laquelle utiliser. D'autant plus qu'au moment o\`{u} le besoin de telle ou telle m\'{e}thode se faisait sentir, je n'avais pas connaissance de son existence. En cons\'{e}quence, je me suis vu trop souvent impl\'{e}menter des m\'{e}thodes d\'{e}j\`{a} existantes, me faisant perdre un temps pr\'{e}cieux.\\
L'exp\'{e}rience accumul\'{e}e durant cette p\'{e}riode m'a permis de ne plus perdre de temps \`{a} r\'{e}-inventer la roue et de me concentrer sur le code ad hoc n\'{e}cessaire \`{a} un test pr\'{e}cis et fiable.

\subsection{M\'{e}thodologie}
L'autre grande perte de temps venait surtout de ma m\'{e}connaissance du produit test\'{e}. Certaines fonctionnalit\'{e}s \'{e}tant tr\`{e}s complexes il m'arrivait de ne pas identifier exactement le probl\`{e}me pour finalement perdre du temps \`{a} impl\'{e}menter du code inefficient.\\
Le meilleur exemple que je puisse citer s'est produit quand j'ai commenc\'{e} \`{a} tester des bugs int\'{e}gr\'{e}s \`{a} des workflows complexes. Un certain nombre de manipulations et/ou de configurations \'{e}tait alors n\'{e}cessaire pour que le bug survienne. Et c'est dans ces conditions que j'ai impl\'{e}ment\'{e} la quasi totalit\'{e} du workflow au niveau SDK pour finalement arriver sur la zone \`{a} tester. Le test fonctionnait bien, il \'{e}chouait sur les versions bugg\'{e}es et passait sur les versions fix\'{e}es. Le probl\`{e}me ne vient pas du test que j'ai impl\'{e}ment\'{e}, car celui-ci faisait ce qu'il fallait, mais de la mani\`{e}re. J'ai impl\'{e}ment\'{e} ce code en approximativement 3 jours. Alors que si j'avais utilis\'{e} le CMS pour g\'{e}n\'{e}rer un document \`{a} une \'{e}tape seulement du bug, et que si j'avais impl\'{e}ment\'{e} uniquement l'appel \`{a} tester, j'aurais pu impl\'{e}menter ce test en quelques heures seulement!

\subsection{Travail en \'{e}quipe}
Le framework de test est entretenu quotidiennement par des dizaines de testeurs impl\'{e}mentant des tests et modifiant le framework lui-m\^{e}me. Le r\'{e}sultat de l'ex\'{e}cution des suites de tests est visible par testeurs, d\'{e}veloppeurs et par d'autres. J'\'{e}tais donc int\'{e}gr\'{e} \`{a} une grande \'{e}quipe, leurs travaux ayant des r\'{e}percussion sur les miens, et les miens sur les leurs. Il faut donc faire attention au code que l'on \textquote{push} sur Perforce! L'anecdote me venant \`{a} l'esprit s'est pass\'{e}e un vendredi soir, alors que je venais de terminer un test et que celui-ci se comportait bien. J'ai livr\'{e} mon test sur perforce. L'ennui \'{e}tait que j'utilisais un package dans lequel je mettais des m\'{e}thodes de test pour ne pas polluer mon test \textquote{final}. Je n'ai malheureusement pas pens\'{e} \`{a} revoir l'import de mes sources. Mon code, compilant correctement en local, ne compilait plus sur le serveur et a donc fait crash\'{e} toute la suite de test. Le mail automatique envoy\'{e} \`{a} inform\'{e} tout le monde de mon erreur de manipulation (illustr\'{e}e annexe \ref{annexe:crashedBuildBecauseOfMe} page \ref{annexe:crashedBuildBecauseOfMe})!\\

J'ai eu aussi la chance de travailler dans une \'{e}quipe maitrisant tr\`{e}s bien le framework et le produit. J'ai appris \'{e}norm\'{e}ment pendant les entretiens que j'ai eu avec ces diff\'{e}rentes personnes.

