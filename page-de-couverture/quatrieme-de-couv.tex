\clearpage
\ifodd\thepage\hbox{}\newpage\else\fi%si page paire ou impaire
\thispagestyle{empty}\parindent=0pt
{\Large \textbf{Titre du r\'{e}sum\'{e}}}

{\large \textbf{sous-titre de mon r\'{e}sum\'{e}}}

\hrulefill%trace un trait horizontal
\begin{center}
{\Large \textbf{R\'{e}sum\'{e}}}
\end{center}
Ici, mon r\'{e}sum\'{e} en français long de 200 mots maximum

\vspace*{\stretch{1}}
\hrulefill%trace un trait horizontal

{\Large \textbf{Mots-clefs}}

\hrulefill%trace un trait horizontal
\begin{enumerate}
\item Software tester
\item JavaEE
\item Agile
\item Jenkins
\end{enumerate}







\vspace*{\stretch{1}}
\selectlanguage{english}
{\Large \textbf{English title}}

{\large \textbf{Sub-title}}

\hrulefill%trace un trait horizontal
\begin{center}
{\Large \textbf{Abstract}}
\end{center}
Here, an 200 words maximum long English abstract

\vspace*{\stretch{1}}
\hrulefill%trace un trait horizontal

{\Large \textbf{Keywords}}

\hrulefill%trace un trait horizontal

\selectlanguage{french}
Ici, une liste de mots-cl\'{e}s

\vspace*{\stretch{1}}
%\hrulefill

%{\Large \textbf{\'Etablissement} }

%\hfill\includegraphics[scale=0.5]{../images/clubdesdevp.png} 

%\hrulefill



