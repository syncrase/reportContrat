\chapter*{Introduction}
\addcontentsline{toc}{chapter}{Introduction}
\markboth{Introduction}{}

Ce présent rapport a pour objet la présentation, d'une part, de SAP, entreprise dans laquelle j'ai fais mon contrat de professionnalisation, et d'autre part, de mes différentes activités effectuées lors de cette période.\\

Mon contrat était divisé en deux périodes, la première s'étendait du 15 juillet au 30 septembre 2014 et la seconde du 15 février au 30 septembre 2015. Ces longues périodes de travail ont permis une immersion complète dans le monde de l'entreprise et surtout dans mes missions.\\


Au commencement de mon contrat de professionnalisation j'ai participé aux tests, raison d'exister de l'équipe \gls{ST} Automation dans laquelle j'ai été intégré. Ainsi, après un certain temps de formation, j'étais capable d'implémenter des tests de qualités et fiables. Aujourd'hui encore ces tests sont exécutés quotidiennement pour s'assurer de la non-régressions du logiciel Web Intelligence.\\
Dans la seconde partie de mon contrat de professionnalisation, je me suis occupé d'implémenter l'accès \`{a} leur logiciel de traçabilité, et ce dans le respect des normes de sécurités prescrites par SAP. Une fois cela terminé et fonctionnel, j'ai entamé le projet qui aura été le plus formateur de mon contrat : le développement d'un plugin pour Jenkins devant se substituer à un autre déjà existant. Développer un plugin pour un logiciel qui m'était inconnu, et aux fonctionnalités qu'alors je découvrais, s'est présenté à moi comme un véritable challenge. Après un certain temps de recherche et de développement, celui-ci a commencé à être utilisé dès lors que ses fonctionnalités égalaient celles du plugin précédant, aujourd'hui désintallé, le mien l'ayant remplacé.\\

Ce temps que j'ai passé dans l'équipe ST Automation à été très formateur et j'ai pu progresser tant en Java que dans ma méthode de travail. Ces différentes missions que je me suis vu effectuer m'ont permises d'étudier le Framework de test de Web Intelligence sous plusieurs angles. Cela m'a permis d'avoir aujourd'hui une vision globale de celui-ci, sans pour autant en comprendre tous les détails, et d'être ainsi plus à même de m'adapter à tout nouveau Framework que je serai appelé à utiliser.